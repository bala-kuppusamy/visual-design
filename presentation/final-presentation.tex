\documentclass[]{article}
\usepackage{lmodern}
\usepackage{amssymb,amsmath}
\usepackage{ifxetex,ifluatex}
\usepackage{fixltx2e} % provides \textsubscript
\ifnum 0\ifxetex 1\fi\ifluatex 1\fi=0 % if pdftex
  \usepackage[T1]{fontenc}
  \usepackage[utf8]{inputenc}
\else % if luatex or xelatex
  \ifxetex
    \usepackage{mathspec}
  \else
    \usepackage{fontspec}
  \fi
  \defaultfontfeatures{Ligatures=TeX,Scale=MatchLowercase}
\fi
% use upquote if available, for straight quotes in verbatim environments
\IfFileExists{upquote.sty}{\usepackage{upquote}}{}
% use microtype if available
\IfFileExists{microtype.sty}{%
\usepackage{microtype}
\UseMicrotypeSet[protrusion]{basicmath} % disable protrusion for tt fonts
}{}
\usepackage[margin=1in]{geometry}
\usepackage{hyperref}
\hypersetup{unicode=true,
            pdftitle={Final Presentation Group 4},
            pdfauthor={Bala Kuppusamy, Minglan Ye, Jiamin Lei (Storm Chasers)},
            pdfborder={0 0 0},
            breaklinks=true}
\urlstyle{same}  % don't use monospace font for urls
\usepackage{graphicx,grffile}
\makeatletter
\def\maxwidth{\ifdim\Gin@nat@width>\linewidth\linewidth\else\Gin@nat@width\fi}
\def\maxheight{\ifdim\Gin@nat@height>\textheight\textheight\else\Gin@nat@height\fi}
\makeatother
% Scale images if necessary, so that they will not overflow the page
% margins by default, and it is still possible to overwrite the defaults
% using explicit options in \includegraphics[width, height, ...]{}
\setkeys{Gin}{width=\maxwidth,height=\maxheight,keepaspectratio}
\IfFileExists{parskip.sty}{%
\usepackage{parskip}
}{% else
\setlength{\parindent}{0pt}
\setlength{\parskip}{6pt plus 2pt minus 1pt}
}
\setlength{\emergencystretch}{3em}  % prevent overfull lines
\providecommand{\tightlist}{%
  \setlength{\itemsep}{0pt}\setlength{\parskip}{0pt}}
\setcounter{secnumdepth}{0}
% Redefines (sub)paragraphs to behave more like sections
\ifx\paragraph\undefined\else
\let\oldparagraph\paragraph
\renewcommand{\paragraph}[1]{\oldparagraph{#1}\mbox{}}
\fi
\ifx\subparagraph\undefined\else
\let\oldsubparagraph\subparagraph
\renewcommand{\subparagraph}[1]{\oldsubparagraph{#1}\mbox{}}
\fi

%%% Use protect on footnotes to avoid problems with footnotes in titles
\let\rmarkdownfootnote\footnote%
\def\footnote{\protect\rmarkdownfootnote}

%%% Change title format to be more compact
\usepackage{titling}

% Create subtitle command for use in maketitle
\newcommand{\subtitle}[1]{
  \posttitle{
    \begin{center}\large#1\end{center}
    }
}

\setlength{\droptitle}{-2em}

  \title{Final Presentation Group 4}
    \pretitle{\vspace{\droptitle}\centering\huge}
  \posttitle{\par}
  \subtitle{Dataset: highschool network}
  \author{Bala Kuppusamy, Minglan Ye, Jiamin Lei (Storm Chasers)}
    \preauthor{\centering\large\emph}
  \postauthor{\par}
      \predate{\centering\large\emph}
  \postdate{\par}
    \date{2019/03/25}


\begin{document}
\maketitle

name: domain-problem class: left, top

\section{Domain Problem}\label{domain-problem}

\begin{itemize}
\item
\end{itemize}

--Datasets contains frendship between students in high school in
December 2013 through serval techiniques.

\begin{itemize}
\tightlist
\item
  Research Question \& Domain Problem
\end{itemize}

--

\begin{itemize}
\tightlist
\item
  Observe which individual is the most popular students in this network.
\end{itemize}

--

\begin{itemize}
\tightlist
\item
  As a shipping routes planner, user needs to track \& forecast storm
  patterns during hurricane seasons to identify safe shipping routes.
\end{itemize}

--

\begin{itemize}
\tightlist
\item
  User needs to analyze the correlation between atmospheric temperature,
  pressure \& storm patterns, if any.
\end{itemize}

--

name: interaction

\section{Encoding / Interaction
Design}\label{encoding-interaction-design}

\begin{itemize}
\tightlist
\item
  Application to be responsive to user inputs and visually plot spatial
  information over map.
\end{itemize}

--

\begin{itemize}
\tightlist
\item
  View overall storm patterns for the selected year and time period.
  Visually identify active storms at any point in time.
\end{itemize}

--

\begin{itemize}
\tightlist
\item
  Track progress of storms on a daily basis \& identify storm location
  \& category.
\end{itemize}

--

\begin{itemize}
\tightlist
\item
  Pinpoint individual storms in the map \& gather details about them.
\end{itemize}

--

\begin{itemize}
\tightlist
\item
  Visually compare year-over-year changes in storm wind speed \& air
  pressure pattern.
\end{itemize}

--

\begin{itemize}
\tightlist
\item
  Compare monthly atmospheric temperatures across years to understand
  correlation between temperature \& storm patterns.
\end{itemize}

name: next-steps

\section{Next Steps / Future Work}\label{next-steps-future-work}

--

\begin{itemize}
\tightlist
\item
  Use predictive algorithms to understand the correlation between
  different atmospheric variables and predict storm paths for future
  storms.
\end{itemize}

--

\begin{itemize}
\tightlist
\item
  Use additional data analysis techniques to explore the data and gain
  insights.
\end{itemize}

--

\begin{itemize}
\tightlist
\item
  Use better visualization options to create more responsive \& faster
  visualizations.
\end{itemize}

\begin{center}\rule{0.5\linewidth}{\linethickness}\end{center}

name: thanks count: false class: center, middle

\section{Thank You!}\label{thank-you}


\end{document}
